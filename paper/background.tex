\section{Background}

\subsection{Program Equivalence}

We use a particularly strict definition of program equivalence. We say
that two \arch{} functions, the target and rewrite, are equivalent
if, when run starting at identical machine states (registers, stack,
heap), one of the following hold:

\begin{itemize}
\item both run to completion, with identical memory state (stack and heap) and identical user-specified output registers (often, but not always, rax)
\item both result in a run-time error
\item both loop forever
\end{itemize}

According to this definition, two \arch{} functions are not equivalent
if they alter the stack differently. This is not a fundamental
assumption of our work, and many authors \todo{cite} demonstrate
that the stack can be assumed not to alias with heap. However,
optimizations that change stack writes can cause buggy programs to
break in unexpected ways. For example, a buggy software might run
correctly as long as a certain function does not modify the stack;
optimizing this function and making use of a stack-allocated temporary
might unintentionally break the program. Thus, to ensure soundness, at
the cost of ruling out some optimizations, we consider the possibility
that heap pointers alias stack locations when proving equivalence.

A consequence of our definition's sensitivity to run-time errors and
infinite loops is that we seek a bisimulation relation, rather than a
simulation relation, to demonstrate equivalence.

\subsection{Bisimulation Relations}\label{sec:proofobl}

\todo{bisimulation relation definition}

Constructing a bisimulation relation for two programs starts by
modeling the two programs as labeled transition systems over a set of
states that capture both data and control flow. A common technique
is to use \emph{cutpoints}. A cutpoint is a pair $(p,q)$ where $p$
is a program point in the target and $q$ a program point in the
rewrite. Each state in the transition system for each program is a
pair $(C, \sigma)$ where $C$ is a cutpoint and $\sigma$ is a machine
state encoding the values of all the registers and memory locations.
The transitions correspond to paths between cutpoints in one of the
programs.

In our work we use a graph structure, called a \emph{\bisimrep}
to describe a bisimulation. This is similar to, but varies from,
similar graphs such as the ``joint transfer function graph"
of~\cite{Dahiya17ASPLAS}. The \bisimrep{} is a directed graph where
each node is a cutpoint. Each edge from cutpoint $C$ relating program
points $(p,q)$ to cutpoint $C'$ relating program points $(p',q')$ is
labeled by a path $P$ from $p$ to $p'$ and a path $Q$ from $q$ to
$q'$. The only loops in the \bisimrep{} are self-loops. We label each
cutpoint $C$ with an invariant $I_C$. This graph induces a relation on
the state transition systems corresponding to the target and rewrite
programs. That is, the states $(C, \sigma)$ and $(C', \sigma')$ are
related if and only if $C = C'$ and $I_C(\sigma,\sigma')$ hold. When
this relation is a bisimulation, we say that the \bisimrep{} is
\emph{correct}.

We can check the correctness of a \bisimrep{} by
verifying the following proof obligations:

\begin{itemize}
\item For each edge labeled by paths $P$ and $Q$ from cutpoint $C =
(p,q)$ to $C' = (p',q')$: if invariant $I_C$ holds for states $\sigma,
\rho$ at $(p,q)$, then executing paths $P$ and $Q$ in the target and rewrite
results in states $\sigma',\rho'$ which satisfy $I_{C'}$.
\item For each cutpoint $C$, \todo{finish this -- it's subtle}
\end{itemize}


\subsection{Modules and $\Z{n}$}

On processors, arithmetic does not happen over $\mathbb{Z}$, but
rather over a space of bitvectors. Mathematically, we can model
an $n$-bit bitvector as an element of $\Z{n}$, that is integers
modulo $2^n$. In the course of our work, we deal with vectors and
matrices over $\Z{n}$ and compute solutions to linear equations over
matrices. However, because the space $\Z{n}$ is not a field, but
rather only a commutative ring, the traditional treatment of linear
algebra does not fully apply. Instead of \emph{vector spaces} we
have \emph{$\Z{n}$-modules}. There are many parallels between vector
spaces and modules, and we need not dwell on them at length. In both
settings, we have the concept of a \emph{basis} or \emph{generating
set} where linear combinations of a set of vectors generate an entire
space. However, a key difference is that in modules we do not have the
concept of linear independence. There is no well-defined notion of
dimension for a module. As a result, when reading about modules, some
of the traditional concepts from linear algebra over fields apply,
but not all do. Instead of a \emph{nullspace}, in the linear algebra
setting, we work with \emph{kernels}. Note that linear algebra over
$\mathbb{Z}$ is easier than linear algebra over $\Z{n}$ because the
integers form a \emph{principal integral domain} (PID), allowing the
construction of the field of fractions $\mathbb{Q}$; however, $\Z{n}$
is not a PID and thus no field of fractions exists. For a systematic
treatment of modules, please see \todo{citation needed}. In our work,
we use SageMath~\cite{sagemath} (version 7.5.1) to perform the needed
computations over $\Z{64}$.

\subsection{Kernel}

\todo{Describe the basic trick of mining linear relationships from
data by computing the kernel of a matrix (i.e. the nullspace)}
