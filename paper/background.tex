\section{Background}

\subsection{Program Equivalence}

We use a particularly strict definition of program equivalence. We say
that two \arch{} functions, the target and rewrite, are equivalent
if, when run starting at identical machine states (registers, stack,
heap), one of the following hold:

\begin{itemize}
\item both run to completion, with identical memory state (stack and heap) and identical output registers (usually, but not always, rax)
\item both result in a run-time error
\item both loop forever
\end{itemize}

According to this definition, two \arch{} functions are not equivalent
if they alter the stack differently. This is not a fundamental
assumption of our work, as many authors \todo{cite} demonstrate
that the stack can be assumed not to alias with heap. However,
optimizations that change the writes to the stack can cause buggy
programs to break in unexpected ways. For example, a buggy software
might run correctly as long as a certain function does not modify
the stack; replacing such a function with an optimized version that
uses the stack to store a temporary might unintentionally break the
program. Thus, to ensure soundness, at the cost of ruling out some
optimizations, we do not assume that heap pointers do not alias the
stack when proving equivalence.

\subsection{Bisimulation Relations}

The mathematical definition of a bisimulation relation specifies
a correspondence between two state transition systems, and in the
equivalence checking literature bisimulation relations are widely
used to prove equivalence of programs. \todo{say a little more here;
perhaps provide definition of bisimulation relation}

The basic technique for building a bisimulation relations between two
programs is to identify a set of \emph{cutpoints}. A cutpoint is a
program point in the target and a corresponding program point in the
rewrite. The states in the transition system of each program consist
of a pair $(C, \sigma)$ where $C$ is a cutpoint and $\sigma$ is a
machine state encoding the values of all the registers and memory
locations. A bisimulation relation is induced by choosing an invariant
$I$ that relates the machine states at the two programs for each
cutpoint. We say that states $(C, \sigma)$ and $(C, \sigma')$ are
related if an only if $I(\sigma, \sigma')$ hold. \todo{cite pldi2000
or Bansal's undef paper}. \todo{explain how bisimulation relations
in the past encoded some assumptions about paths that reach program
points}

\todo{explain the graph structure we use}

\subsection{Modules and $\Z{n}$}

On processors, arithmetic does not happen over $\mathbb{Z}$, but
rather over a space of bitvectors. Mathematically, we can model
an $n$-bit bitvector as an element of $\Z{n}$, that is integers
modulo $2^n$. In the course of our work, we deal with vectors and
matrices over $\Z{n}$ and compute solutions to linear equations over
matrices. However, because the space $\Z{n}$ is not a field, but
rather only a commutative ring, the traditional treatment of linear
algebra does not fully apply. Instead of \emph{vector spaces} we
have \emph{$\Z{n}$-modules}. There are many parallels between vector
spaces and modules, and we need not dwell on them at length. In both
settings, we have the concept of a \emph{basis} or \emph{generating
set} where linear combinations of a set of vectors generate an
entire space. However, a key difference is that in modules we do
not have the concept of linear independence. Rather, there is no
well-defined notion of dimension for a module. As a result, when
reading about modules, some of the traditional concepts from linear
algebra over fields apply, but not all do. Note that linear algebra
over $\mathbb{Z}$ is easier than linear algebra over $\Z{n}$ because
the integers form a \emph{principal integral domain} (PID), allowing
the construction of the field of fractions $\mathbb{Q}$; however,
$\Z{n}$ is not a PID and thus no field of fractions exists. For a
systematic treatment of modules, please see \todo{citation needed}. In
our work, we use SageMath~\cite{sagemath} (version 7.5.1)
to perform the needed computations over $\Z{64}$.

