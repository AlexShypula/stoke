\section{Evaluation}

In this section we seek to answer the following questions,

\begin{itemize}
\item For a representative collection of vectorization optimizations, what proportion can we verify?
\item How does our \bisim{} inference algorithm compare with other state-of-the-art techniques?
\item How efficient is our algorithm to enumerate candidate \bisimreps{} for verification?
\item Are the complexities of the algorithm necessary
        - using equivalence classes
        - how important / useful are the sanity checks
  %% IDEA: look in POD and see if branch conditions are equal?
\end{itemize}

\subsection{Benchmarks}

To evaluate \tool{} we construct a set of benchmarks for verifying
vectorization optimizations. We started with the \emph{Test Suite
for Vectorizing Compilers} (TSVC), which was developed ``to assess
the vectorizing capabilities of compilers'' and ported to C in
\cite{TSVC}. These benchmarks operated over arrays of floating point
values of fixed size. However, because modern SMT solvers are limited
in their ability to reason accurately about floats and our desire to
demonstrate proving equivalence over unbounded programs, we rewrote
the benchmarks to operate over integer arrays and parameterized the
input length.  We had to remove three classes of benchmarks from the
original TSVC set:

\begin{itemize}
\item Benchmarks that could not be vectorized with either GCC 4.9.2 nor clang 3.4 using -msse4.2 and -O3.\todo{number?}
\item Duplicate benchmarks where nearly identical assembly code was generated after compilation as another benchmark. \todo{number?}
\item Benchmark functions that contained calls to other methods. \todo{number?}
\item Benchmarks containing doubly-nested loops.  These often triggered aggressive loop tiling optimizations beyond the scope of our work. \todo{number?}
\end{itemize}

We had to remove duplicate benchmarks because some of the TSVC
benchmarks were designed to see if a compiler could perform an
analysis to verify an optimization was safe to perform; however,
after compilation, the two programs would be the same. 26 benchmarks
remained. Most of the benchmarks iterate over one or more arrays (up
to 5), perform some computation (often adds and multiplies), and
update the arrays. Some benchmarks process the array forwards, some
backwards, and some with a stride.

\subsection{Discharging Proof Obligations}

We constructed test cases by creating machine states containing input
vectors of randomly-chosen bytes. This same set of test cases was
sufficient to obtain code coverage over all these benchmarks. We
also manually placed a bound on the input size to exclude overflow
conditions and other undefined behavior; this is generally needed for
proving equivalence~\cite{Dahiya17HVC}.

\todo{discuss use of ARM and flat memory model}

\todo{describe cloud infrastructure}

\todo{add table of results}

\todo{discuss failures and timeouts}

\todo{add comparison with recent work}

\todo{add straw-man comparison; try using wildcard equivalence class with bound 1}


%% TODO: add table with 
%   - benchmark
%   - ASM LOC O1 / O3
%   - equivalent / non-equivalent / timeout
%   - linear algebra time
%   - other enumeration time
%   - total time for our algorithm
%   - total CPU-time for proof obligation discharge

\subsection{Results}

\subsection{strlen case study}

GNU Lib C contains widely-used functions, such as strlen, which are
used frequently throughout userspace applications. In many cases,
compilers are unable to vectorize performance-critical functions, and
instead programmers perform the vectorization themselves. This is the
case for the strlen function since libc \todo{version number}. The
handwritten C code uses clever bit-manipulation techniques to read
up to 8 characters from the string at a time, check if any character
contains a null byte, and to then either continue or terminate.

To the best of our knowledge, no other technique in the equivalence
checking or translation validation literature is able to handle this
kind of example. 
\todo{not sure what we'll put in the example yet, so I don't want to fill in too much here}
