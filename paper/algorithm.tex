\section{Algorithm}

\todo{Write a short summary of the overall idea}

The enumeration of candidate \bisimreps{} proceeds in four main steps.
First, we identify linear relationships between the number of times
basic blocks are executed in the target and the rewrite; these
relationships help identify a correspondence between the loops
of the two programs (Section \ref{algstep1}). Second, we infer
a $\Z{64}$-module of flow-insensitive invariants that hold at a
\cutpoint, regardless of the paths taken in each program to reach the
\cutpoint{} (Section \ref{algstep2}). Third, we compute a submodule
of flow-sensitive loop invariants, meaning invariants that appear
to hold inductively over loop iterations (Section \ref{algstep3}).
Lastly, we use the flow-sensitive loop invariants to create a search
space of \bisimreps. (Section \ref{algstep4}). We enumerate \bisimreps and
apply inexpensive checks to eliminate those that are invalid (Section
\ref{algstep5}), followed by a more expensive verification that a
remaining \bisimrep{} is correct (Section \ref{algstep6}).

\subsection{Representing Paths as Vectors}

We use control flow data from dynamic execution traces to learn
a \bisim{} between the target and rewrite programs. A recurring
technique in our approach is to count the number of times each basic
block is executed on a pair of execution paths (one through the target
and one through the rewrite). We instrument each basic block with an
extra instruction that counts the number times the block executes.
In block $i$ of the target we use \shadow{} $n_i$ to represent its
execution count, and $n_i'$ for basic block $i$ of the rewrite.

It is often convenient to represent these counts as a vector. Given
a path $P$ through the target and a path $Q$ through the rewrite, we
define the \emph{vector representation of paths P,Q} or $v(P,Q)$ as
the vector that encodes the number of times each basic block appears
on the path as follows:

$$v(P,Q) = <n_0, n_1, \dotsc, n_k, n_0', n_1', \dotsc, n_{k'}'>$$

\todo{do we need to say more here or illustrate some properties of
this representation? e.g. n0 = 1 when starting at program entry}

\subsection{Learning Linear Control Flow Relationships} \label{algstep1}

In this step our goal is to learn relationships between the number of
times basic blocks execute and to identify corresponding loops in the
target and rewrite. For example, we might observe that a loop body in
the target executes four times as often as a corresponding loop body
in the rewrite.

To learn the control flow relationships, we construct a
matrix $\controlflowmatrix$ and compute its nullspace,
$\nullsp{\controlflowmatrix}$. We run each test case $\tau_i$ on
the target and rewrite programs. Each $\tau_i$ exhibits paths $P_i$
and $Q_i$. We construct $\controlflowmatrix$ by setting the $i$th
row to $v(P_i, Q_i)$. Each vector $w$ in the generating set of
$\nullsp{\controlflowmatrix}$ satisfies $w \cdot v(P_i, Q_i) = 0$
for each $i$. In essence, each $w$ corresponds to a linear equality
between basic block counts; for example, one such $w$ might correspond
to the linear equality $n_0 = n_0'$, which states that the entry block
in the target is executed the same number of times as the entry block
through the rewrite. The consequence is that given complete paths $P$
and $Q$ through the target and the rewrite, we can test whether such
paths are consistent with the provided test cases by computing $w\cdot
v(P,Q)$ for each $w$ and checking if the results are 0. If so, then
$P$ and $Q$ are consistent; if not, then $P$ and $Q$ are inconsistent
with the test data.

\note{There might be a nicer linear algebra approach to do this next
bit} Armed with these linear equalities, we perform a search to
identify corresponding loops. A naive approach is to check if any
equality corresponding to a generator of $\nullsp{\controlflowmatrix}$
is of the form $n_i = k*n_j'$; however, this will not reliably work
as the generating set may instead include a relationship of the form
$n_i + n_0 = k*n_j' + n_0'$ and $n_0 = n_0'$ separately. These two
linear equalities both imply $n_i = k*n_j'$, but the information
cannot be recovered by pattern matching on the linear equalities.
Instead, we enumerate strongly connected components $C_1, C_2, \dotsc$
and $C_1', C_2', \dotsc$ of the control flow graphs of the target and
rewrite. For each $C_i$ we pick a basic block to serve as the target's
half of the \cutpoint{} and enumerate paths $SelfLoop(C_i)$ from the
\cutpoint{} to itself up to some bound \todo{we haven't explained
cutpoints yet}. For each $C_i$ and $C_j'$ we consider $SelfLoops =
SelfLoop(C_i) \times SelfLoop(C_j')$ and check to see if there exists
$x \in SelfLoops$ where $v(x) \in \nullsp{\controlflowmatrix}$. If
so, we have identified a corresponding pair of paths in the target
and rewrite programs. If no pairs are found, we double the bound. In
practice, we initialize the bound to 1 and allow a maximum of 128.

We use a search to choose the \cutpoint{} for each pair of
corresponding loops. In our early experiments, merely choosing the
loop heads or loop exits of corresponding loops was insufficient;
depending on the choice of program points, different values are live,
which affect the quality of the invariants ultimately learned. For
each pair of corresponding loops, we perform a search over every
possible pair of program points that could be used as a \cutpoint.
For each candidate cutpoint, we perform the above correspondence of
loops, the invariant learning described in section \ref{algstep2} and
section \ref{algstep3} and use a heuristic to evaluate the quality of
the learned invariants. The heuristic is to count the number of live
variables constrained by learned invariants; more is better. We then
use the \cutpoint{} that generated the best invariant.

Another option might be to preserve the values of
clobbered registers in a \shadow{} and use these for computing
invariants as well; however, this would increase the computational
cost of invariant learning.

\subsection{Learning Linear Flow-Insensitive Loop Invariants} \label{algstep2}

For each test case $\tau_i$ we run the target and the rewrite
programs. At each selected program point of each program we record the
register and \shadow{} values. This gives us a list of target data
points $\alpha_1, \alpha_2, \dotsc, \alpha_N$ and a list of rewrite
data points $\beta_1, \beta_2, \dotsc, \beta_M$. For each $1 \leq i
\leq N, 1 \leq j \leq M$, we concatenate $\alpha_i$ and $\beta_j$
and add this as a row in our matrix. We compute the nullspace of
this matrix, and this yields a generating set for a $\Z{64}$-module of
flow-insensitive invariants; we can take sums, differences, and scalar
multiples of the generating set elements, and they will still hold across all
the provided data.

\subsection{Identifying the Subspace of Linear Flow-Sensitive Loop Invariants} \label{algstep3}

Suppose at a \cutpoint{} the generating set for the flow-insensitive invariants are as follows:

\begin{eqnarray*}
a_{11}r_1 + a_{12}r_2 + \dotsm + a_{1n}r_n &=& b_{11}n_1 + b_{12}n_2 + \dotsm + b_{1m}n_m + c_1\\
a_{21}r_1 + a_{22}r_2 + \dotsm + a_{2n}r_n &=& b_{21}n_1 + b_{22}n_2 + \dotsm + b_{2m}n_m + c_2\\
\vdots &\vdots& \vdots\\
a_{k1}r_1 + a_{k2}r_2 + \dotsm + a_{kn}r_n &=& b_{k1}n_1 + b_{k2}n_2 + \dotsm + b_{km}n_m + c_k
\end{eqnarray*}

Note that, in general, these invariants would be very difficult to
prove because proving them would require reasoning about all paths
a program could take to arrive at a \cutpoint. As a result, proving
all of these invariants directly would be as hard as reasoning about
each program individually, and then asserting equality at the end.
Moreover, proving any one of these invariants might be very difficult,
because each one may capture aspects of a complete execution run of
one program or the other. \note{this will be more clear following the example}

Instead, our goal is to create an explicit \bisim{} where a submodule
of these invariants hold where two traces align at a \cutpoint.
Suppose at this \cutpoint{} there is self-loop with paths $P, Q$.
If the dot product of $v(P,Q)$ with the corresponding vector from
the $i$th equation is non-zero, then this invariant cannot easily be
proven inductively only considering this loop. Note that here the
constant term of the invariant is ignored. \todo{explain this better}

Now, we consider the paths the programs take between \cutpoints.

\subsection{Enumerating Search Space} \label{algstep4}

Given the set of \cutpoints and{} a space of flow-sensitive loop
invariants, we can add all the edges needed to complete the \bisimrep.
We choose a topological ordering on the \cutpoints{} $C_1, C_2,
\dotsc, C_n$ and systematically learn edges from $C_i$ to $C_j$ with
$i < j$. We use the flow-sensitive loop invariants to partition all
pairs of paths into $C_i$ into equivalence classes. For class $C_i$
we represent each equivalence class with a vector over $\Z{64}$,
with one entry per flow-insensitive invariant in the generating set.
The $i$th entry the vector corresponds to the difference between
the unspecified constant term of the corresponding invariant. For
a pair of paths $P, Q$ from $C_1$ to $C_i$ we take $v(P,Q)$ dotted
with the invariant to determine the equivalence class. To prevent
an exponential blowup in the number of paths to consider, when we
consider pairs of paths from $C_j$ to $C_i$ we arbitrarily choose one
representative path from $C_1$ to $C_j$ to represent the prefixes of
all paths from $C_j$ to $C_i$.  

To enumerate \bisimreps{} we consider every possible assignment of an
equivalence class to each \cutpoint. For each assignment, we enumerate
paths up to a bound from $C_i$ to $C_j$ where $i \leq j$ and pick only
the edges whose equivalence class matches that of the \cutpoint. We
add these edges to the \bisimrep. Additionally, sometimes its helpful
to consider only a subset of the invariants for the equivalence
classes, so we also search over all the subsets.

\subsection{Eliminating Spurious Candidates} \label{algstep5}

We perform quick checks on the \bisimrep{} before attempting to prove
it correct. First, we ensure that the graph is connected; in some
cases, the choice of equivalence class will preclude the existence of
edges from a node to the exit, and thus such an equivalence class is
invalid.

Second, we use the data from the test cases to ensure that the
\bisimrep{} is usable. We want to verify that, if test case $\tau$
executes on paths $P$ and $Q$ through the rewrite, then the paths
may be decomposed into $P = P_1P_2\dotsm P_k$ and $Q = Q_1Q_2\dotsm
Q_k$ such that each $(P_i, Q_i)$ is an edge in the \bisimrep. To do
so, we treat the \bisimrep as the graph of a nondeterministic finite
automata whose transitions are labeled by pairs of program paths, and
check that this automata accepts the input $(P, Q)$. 

Following each transition of the automata, we record the program state
corresponding to the execution of paths $P_1\dotsm P_i$ and $Q_1
\dotsm Q_i$, which is data for some cutpoint $C$. We additionally
check that, at each transition, the heap of the target matches the
heap of the rewrite; if not, we know that we cannot prove the programs
equivalent.

\subsection{Learning Non-Linear Invariants and Verifying Correctness} \label{algstep6}

\begin{figure}
\begin{eqnarray*}
Inv &\rightarrow& \sum c_iv_i = c \hspace{2mm}\bigg|\hspace{2mm} \pm v_1 \pm v_2 \leq c \hspace{2mm}\bigg|\hspace{2mm} m_1 = m_2 \hspace{2mm}\bigg|\hspace{2mm} m = v\\
v   &\rightarrow& Reg \hspace{2mm}\bigg|\hspace{2mm}  Reg[i:j]  \hspace{2mm}\bigg|\hspace{2mm} n_i\\
\end{eqnarray*}
\caption{The language of invariants.  Each $c$ represents a bitvector constant, $m$ represents a memory dereference, $n_i$ a count of basic block executions.  $Reg$ represents an \arch{} register and $Reg[i:j]$ represents a subregister.}
\end{figure}
