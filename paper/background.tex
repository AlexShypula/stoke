\section{Background}

\subsection{Bisimulation Relations}

The mathematical definition of a bisimulation relation specifies
a correspondence between two state transition systems, and in the
equivalence checking literature bisimulation relations are widely
used to prove equivalence of programs. Many works model the program as
a state transition system by treating one state as the combination of
a program point with all live variables. In our setting, this model
isn't sufficient; relating the machine state and current program point
is not enough to prove equivalence in some examples because it forgets
the past history of the execution which can encode vital knowledge

\subsection{Modules and $\Z{n}$}

On processors, arithmetic does not happen over $\mathbb{Z}$, but
rather over a space of bitvectors. Mathematically, we can model
an $n$-bit bitvector as an element of $\Z{n}$, that is integers
modulo $2^n$. In the course of our work, we deal with vectors and
matrices over $\Z{n}$ and compute solutions to linear equations over
matrices. However, because the space $\Z{n}$ is not a field, but
rather only a commutative ring, the traditional treatment of linear
algebra does not fully apply. Instead of \emph{vector spaces} we
have \emph{$\Z{n}$-modules}. There are many parallels between vector
spaces and modules, and we need not dwell on them at length. In both
settings, we have the concept of a \emph{basis} or \emph{generating
set} where linear combinations of a set of vectors generate an
entire space. However, a key difference is that in modules we do
not have the concept of linear independence. Rather, there is no
well-defined notion of dimension for a module. As a result, when
reading about modules, some of the traditional concepts from linear
algebra over fields apply, but not all do. Note that linear algebra
over $\mathbb{Z}$ is easier than linear algebra over $\Z{64}$ because
the integers form a \emph{principal integral domain} (PID), allowing
the construction of the field of fractions $\mathbb{Q}$; however,
$\Z{n}$ is not a PID and thus no field of fractions exists. For a
systematic treatment of modules, please see \todo{citation needed}.


